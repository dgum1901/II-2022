% Options for packages loaded elsewhere
\PassOptionsToPackage{unicode}{hyperref}
\PassOptionsToPackage{hyphens}{url}
%
\documentclass[
  oneside]{memoir}
\usepackage{amsmath,amssymb}
\usepackage{lmodern}
\usepackage{iftex}
\ifPDFTeX
  \usepackage[T1]{fontenc}
  \usepackage[utf8]{inputenc}
  \usepackage{textcomp} % provide euro and other symbols
\else % if luatex or xetex
  \usepackage{unicode-math}
  \defaultfontfeatures{Scale=MatchLowercase}
  \defaultfontfeatures[\rmfamily]{Ligatures=TeX,Scale=1}
\fi
% Use upquote if available, for straight quotes in verbatim environments
\IfFileExists{upquote.sty}{\usepackage{upquote}}{}
\IfFileExists{microtype.sty}{% use microtype if available
  \usepackage[]{microtype}
  \UseMicrotypeSet[protrusion]{basicmath} % disable protrusion for tt fonts
}{}
\makeatletter
\@ifundefined{KOMAClassName}{% if non-KOMA class
  \IfFileExists{parskip.sty}{%
    \usepackage{parskip}
  }{% else
    \setlength{\parindent}{0pt}
    \setlength{\parskip}{6pt plus 2pt minus 1pt}}
}{% if KOMA class
  \KOMAoptions{parskip=half}}
\makeatother
\usepackage{xcolor}
\usepackage{graphicx}
\makeatletter
\def\maxwidth{\ifdim\Gin@nat@width>\linewidth\linewidth\else\Gin@nat@width\fi}
\def\maxheight{\ifdim\Gin@nat@height>\textheight\textheight\else\Gin@nat@height\fi}
\makeatother
% Scale images if necessary, so that they will not overflow the page
% margins by default, and it is still possible to overwrite the defaults
% using explicit options in \includegraphics[width, height, ...]{}
\setkeys{Gin}{width=\maxwidth,height=\maxheight,keepaspectratio}
% Set default figure placement to htbp
\makeatletter
\def\fps@figure{htbp}
\makeatother
\setlength{\emergencystretch}{3em} % prevent overfull lines
\providecommand{\tightlist}{%
  \setlength{\itemsep}{0pt}\setlength{\parskip}{0pt}}
\setcounter{secnumdepth}{-\maxdimen} % remove section numbering
\usepackage[left=15mm, inner=3cm, outer=3cm, top=3cm,bottom=3cm ]{geometry}
\usepackage{mathpazo}
\usepackage[spanish]{babel}
\usepackage[utf8]{inputenc}
\usepackage{graphics,graphicx}
\usepackage{wallpaper}
\usepackage{multicol}
\usepackage{fancyhdr}
\nonzeroparskip
\setlength{\parindent}{12pt}
\setlength{\baselineskip}{5mm}
\pagestyle{fancy}
%\usepackage{parskip}
%\setcounter{secnumdepth}{1}

\renewcommand\thesection{\arabic{section}}
\renewcommand\footrulewidth{0.4pt} % tamanno de la regla abajo

\copypagestyle{headerwchap}{myheadings}
\chapterstyle{bianchi}
\makeevenhead{headerwchap}{\thepage}{}{}
\makeoddhead{headerwchap}{}{}{\thepage}
\makeoddfoot{headerwchap}{}{}{}
\makeevenfoot{headerwchap}{}{}{}
\makeheadrule{headerwchap}{\textwidth}{0.5pt}



\usepackage[ %
style=authoryear, % 
natbib = true, %
dashed = false, %
firstinits = true, 
url =false, %
isbn = false, %
backend = bibtex]{biblatex}
\addbibresource{referencias_proyecto.bib}
%%%%%%%%%%%%%%%%%%%%%%%%%%%%%%%%%%%

%%%%%%%%%%%%%%%%%%%%%%%%%%%%%%%%%%%
%%%% EDITAR SOLO EL SEGUNDO %%%%%%%
%%%% CONJUNTO DE PARENTESIS %%%%%%%
%%%%%%%%%%%%%%%%%%%%%%%%%%%%%%%%%%%

% Titulo del proyecto
\newcommand{\TituloProy}{Ac\'a se pone un buen t\'itulo}  % ACA SE CAMBIA EL TITULO
 
% Nombre de Estudiantes y Carnet
\newcommand{\EstudianteUno}{Maria Jos\'e Corea}
\newcommand{\EstudianteDos}{Cassandra Ram\'irez}
\newcommand{\EstudianteTres}{Daniel Ulate}

\newcommand{\ds}[1]{\ensuremath{\displaystyle{#1}}}


\def\bitcoinB{\leavevmode
  {\setbox0=\hbox{\textsf{C}}%
    \dimen0\ht0 \advance\dimen0 0.2ex
    \ooalign{\hfil \box0\hfil\cr
      \hfil\vrule height \dimen0 depth.2ex\hfil\cr
    }%
  }%
}
\usepackage{booktabs}
\usepackage{longtable}
\usepackage{array}
\usepackage{multirow}
\usepackage{wrapfig}
\usepackage{float}
\usepackage{colortbl}
\usepackage{pdflscape}
\usepackage{tabu}
\usepackage{threeparttable}
\usepackage{threeparttablex}
\usepackage[normalem]{ulem}
\usepackage{makecell}
\usepackage{xcolor}
\ifLuaTeX
  \usepackage{selnolig}  % disable illegal ligatures
\fi
\IfFileExists{bookmark.sty}{\usepackage{bookmark}}{\usepackage{hyperref}}
\IfFileExists{xurl.sty}{\usepackage{xurl}}{} % add URL line breaks if available
\urlstyle{same} % disable monospaced font for URLs
\hypersetup{
  hidelinks,
  pdfcreator={LaTeX via pandoc}}

\author{}
\date{\vspace{-2.5em}}

\begin{document}
\frontmatter

\begin{titlingpage}
  \newcommand{\HRule}{\rule{\linewidth}{0.5mm}} % Defines a new command for the horizontal lines, change thickness here

  \begin{center}

    {\Huge \textbf{UNIVERSIDAD DE COSTA RICA}}\\[0.2cm] 
    {\Large ESCUELA DE MATEM\'ATICA\\[0.2cm]
DEPARTAMENTO DE MATEMÁTICA PURA Y CIENCIAS ACTUARIALES\\[0.2cm] 
MODELOS LINEALES\\[0cm]
} %

   

    \HRule \\[0.6cm]
    \Large \textbf{\TituloProy}
    \HRule \\[0.6cm]

    \vfill
      
    {\Large{\textbf{Bit\'acora I}}}
   
    \vfill
    \begin{center}
      Realizado por %
      \linebreak %
      \linebreak %
      \begin{tabular}{c}
        \hline  \\
        \EstudianteUno  \\ \\
        \EstudianteDos  \\ \\
        \EstudianteTres  \\ \\
        \hline 
      \end{tabular}
    \end{center}

    \vfill
    

    

    % Bottom of the page
    \begin{center}
    \begin{tabular}{l @{\hskip 2in} r}
    \includegraphics[height=2cm]{./images/ucr_marca_de_agua}
  & \includegraphics[height=4cm]{./images/EMat_escuela_matematica_horizontal_1} 
    \end{tabular}
    \end{center}

 %\\[0.1in]
%     \Large{Escuela de Matemática}\\
%     \Large{\textsc{Universidad de Costa Rica}}\\


    \ThisURCornerWallPaper{0.5}{./images/granosgirasol.png}
 
  \end{center}

\end{titlingpage}

%%% Local Variables:
%%% mode: latex
%%% TeX-master: "plantilla_proyecto"
%%% End:

\mainmatter
\fancyhead[LE,RO]{\slshape \rightmark}
\fancyhead[LO,RE]{\slshape \leftmark}
\lhead{CA-404 Modelos Lineales}
\rhead{\thepage}
\fancyfoot[C]{\thepage}

\tableofcontents*

\newpage
\chapter{Bitácora 1}

Se conoce a los mercados inmobiliarios como el conjunto de las acciones
de oferta y demanda de bienes inmuebles, no obstante, la promoción
inmobilaria, la inversión que se puede realizar por instituciones y
compañias, así como individuos, y por último la financiación, la
adquisión inmobilaria.

El presente proyecto de investigación tiene como objetivo analizar con
herramientas estadísticas una base de datos cerrada y contextualizada en
los Estados Unidos de América, analizar y predicir hacia donde se están
movilizando los precios de las propiedades. La base de datos fue
encontrada en Kaggle. Esta base toma en cuenta observaciones desde el 2
de mayo del 2014, hasta el 10 de julio del mismo año, sin embargo, la
base fue publicada en el 2018. En total se encuentran 4600
observaciones.

Sobre la base de datos, la población de estudio está determinada las
casas ubicadas en ciudades de los Estados Unidos de América. La muestra
de estudio es el conjunto de datos que contiene una serie de
características que describen los parámetros de las casas ubicadas en
las distintas ciudades de los Estados Unidos. Es decir, la muestra se
refiere al comportamiento del precio de las propiedades. Ante lo
anterior la unidad elemental de estudio son los datos que favorecen el
precio de la propiedad inmobiliaria, observados de manera diaria por los
dos meses de estudio.

El presente compilado de datos posee algunas variables que se procede a
nombrar y explicar:

\begin{itemize}
\item \texit{date:} La fecha en la cual se obtuvo los datos.
\item \texit{price:} El precio de la casa, expresado en dólares.
\item \texit{bedrooms:} Número de dormitorios.
\item \texit{bathrooms:} Número de baños
\item \texit{sqft_living:} Superficie habitable.
\item \texit{sqft_lot:} Tamaño del lote.
\item \texit{floors:} Número de pisos, entrepisos aceptables.
\item \texit{waterfront:} Vista al mar, característica dicotómica.
\item \texit{view:} Vista, en la escala del 0 al 4.
\item \texit{condition:} Condición de la casa. en la escala del 1 al 5.
\item \texit{sqft_above:} Supercie de la casa.
\item \texit{sqft_basement:} Superficie del sótano, en caso de que posean uno.
\item \texit{yr_built:} Año de construcción.
\item \texit{yr_renovation:} Año de renovación, en caso de que hubiese.
\item \texit{street:} Dirección 
\item \texit{city:} Ciudad 
\item \texit{staetzip:} Código postal.
\item \texit{country:} País.
\end{itemize}

*\textbf{Las características del área se expresan en pies cuadrados.}

\section{Objetivos Específicos:}
 \begin{itemize} 
 \item Dormir

\item Culiar \citep{pinga}

\end{itemize}

\printbibliography

\backmatter
\end{document}
