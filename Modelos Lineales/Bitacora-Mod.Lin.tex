% Options for packages loaded elsewhere
\PassOptionsToPackage{unicode}{hyperref}
\PassOptionsToPackage{hyphens}{url}
%
\documentclass[
  oneside]{memoir}
\usepackage{amsmath,amssymb}
\usepackage{lmodern}
\usepackage{ifxetex,ifluatex}
\ifnum 0\ifxetex 1\fi\ifluatex 1\fi=0 % if pdftex
  \usepackage[T1]{fontenc}
  \usepackage[utf8]{inputenc}
  \usepackage{textcomp} % provide euro and other symbols
\else % if luatex or xetex
  \usepackage{unicode-math}
  \defaultfontfeatures{Scale=MatchLowercase}
  \defaultfontfeatures[\rmfamily]{Ligatures=TeX,Scale=1}
\fi
% Use upquote if available, for straight quotes in verbatim environments
\IfFileExists{upquote.sty}{\usepackage{upquote}}{}
\IfFileExists{microtype.sty}{% use microtype if available
  \usepackage[]{microtype}
  \UseMicrotypeSet[protrusion]{basicmath} % disable protrusion for tt fonts
}{}
\makeatletter
\@ifundefined{KOMAClassName}{% if non-KOMA class
  \IfFileExists{parskip.sty}{%
    \usepackage{parskip}
  }{% else
    \setlength{\parindent}{0pt}
    \setlength{\parskip}{6pt plus 2pt minus 1pt}}
}{% if KOMA class
  \KOMAoptions{parskip=half}}
\makeatother
\usepackage{xcolor}
\IfFileExists{xurl.sty}{\usepackage{xurl}}{} % add URL line breaks if available
\IfFileExists{bookmark.sty}{\usepackage{bookmark}}{\usepackage{hyperref}}
\hypersetup{
  hidelinks,
  pdfcreator={LaTeX via pandoc}}
\urlstyle{same} % disable monospaced font for URLs
\usepackage{graphicx}
\makeatletter
\def\maxwidth{\ifdim\Gin@nat@width>\linewidth\linewidth\else\Gin@nat@width\fi}
\def\maxheight{\ifdim\Gin@nat@height>\textheight\textheight\else\Gin@nat@height\fi}
\makeatother
% Scale images if necessary, so that they will not overflow the page
% margins by default, and it is still possible to overwrite the defaults
% using explicit options in \includegraphics[width, height, ...]{}
\setkeys{Gin}{width=\maxwidth,height=\maxheight,keepaspectratio}
% Set default figure placement to htbp
\makeatletter
\def\fps@figure{htbp}
\makeatother
\setlength{\emergencystretch}{3em} % prevent overfull lines
\providecommand{\tightlist}{%
  \setlength{\itemsep}{0pt}\setlength{\parskip}{0pt}}
\setcounter{secnumdepth}{-\maxdimen} % remove section numbering
\usepackage[left=15mm, inner=3cm, outer=3cm, top=3cm,bottom=3cm ]{geometry}
\usepackage{mathpazo}
\usepackage[spanish]{babel}
\usepackage[utf8]{inputenc}
\usepackage{graphics,graphicx}
\usepackage{wallpaper}
\usepackage{multicol}
\usepackage{fancyhdr}
\nonzeroparskip
\setlength{\parindent}{12pt}
\setlength{\baselineskip}{5mm}
\pagestyle{fancy}
%\usepackage{parskip}
%\setcounter{secnumdepth}{1}

\renewcommand\thesection{\arabic{section}}
\renewcommand\footrulewidth{0.4pt} % tamanno de la regla abajo

\copypagestyle{headerwchap}{myheadings}
\chapterstyle{bianchi}
\makeevenhead{headerwchap}{\thepage}{}{}
\makeoddhead{headerwchap}{}{}{\thepage}
\makeoddfoot{headerwchap}{}{}{}
\makeevenfoot{headerwchap}{}{}{}
\makeheadrule{headerwchap}{\textwidth}{0.5pt}



\usepackage[ %
style=authoryear, % 
natbib = true, %
dashed = false, %
firstinits = true, 
url =false, %
isbn = false, %
backend = bibtex]{biblatex}
\addbibresource{referencias_proyecto.bib}
%%%%%%%%%%%%%%%%%%%%%%%%%%%%%%%%%%%

%%%%%%%%%%%%%%%%%%%%%%%%%%%%%%%%%%%
%%%% EDITAR SOLO EL SEGUNDO %%%%%%%
%%%% CONJUNTO DE PARENTESIS %%%%%%%
%%%%%%%%%%%%%%%%%%%%%%%%%%%%%%%%%%%

% Titulo del proyecto
\newcommand{\TituloProy}{Ac\'a se pone un buen t\'itulo}  % ACA SE CAMBIA EL TITULO
 
% Nombre de Estudiantes y Carnet
\newcommand{\EstudianteUno}{Maria Jos\'e Corea}
\newcommand{\EstudianteDos}{Cassandra Ram\'irez}
\newcommand{\EstudianteTres}{Daniel Ulate}

\newcommand{\ds}[1]{\ensuremath{\displaystyle{#1}}}


\def\bitcoinB{\leavevmode
  {\setbox0=\hbox{\textsf{C}}%
    \dimen0\ht0 \advance\dimen0 0.2ex
    \ooalign{\hfil \box0\hfil\cr
      \hfil\vrule height \dimen0 depth.2ex\hfil\cr
    }%
  }%
}
\usepackage{booktabs}
\usepackage{longtable}
\usepackage{array}
\usepackage{multirow}
\usepackage{wrapfig}
\usepackage{float}
\usepackage{colortbl}
\usepackage{pdflscape}
\usepackage{tabu}
\usepackage{threeparttable}
\usepackage{threeparttablex}
\usepackage[normalem]{ulem}
\usepackage{makecell}
\usepackage{xcolor}
\ifluatex
  \usepackage{selnolig}  % disable illegal ligatures
\fi

\author{}
\date{\vspace{-2.5em}}

\begin{document}
\frontmatter

\begin{titlingpage}
  \newcommand{\HRule}{\rule{\linewidth}{0.5mm}} % Defines a new command for the horizontal lines, change thickness here

  \begin{center}

    {\Huge \textbf{UNIVERSIDAD DE COSTA RICA}}\\[0.2cm] 
    {\Large ESCUELA DE MATEM\'ATICA\\[0.2cm]
DEPARTAMENTO DE MATEMÁTICA PURA Y CIENCIAS ACTUARIALES\\[0.2cm] 
MODELOS LINEALES\\[0cm]
} %

   

    \HRule \\[0.6cm]
    \Large \textbf{\TituloProy}
    \HRule \\[0.6cm]

    \vfill
      
    {\Large{\textbf{Bit\'acora I}}}
   
    \vfill
    \begin{center}
      Realizado por %
      \linebreak %
      \linebreak %
      \begin{tabular}{c}
        \hline  \\
        \EstudianteUno  \\ \\
        \EstudianteDos  \\ \\
        \EstudianteTres  \\ \\
        \hline 
      \end{tabular}
    \end{center}

    \vfill
    

    

    % Bottom of the page
    \begin{center}
    \begin{tabular}{l @{\hskip 2in} r}
    \includegraphics[height=2cm]{./images/ucr_marca_de_agua}
  & \includegraphics[height=4cm]{./images/EMat_escuela_matematica_horizontal_1} 
    \end{tabular}
    \end{center}

 %\\[0.1in]
%     \Large{Escuela de Matemática}\\
%     \Large{\textsc{Universidad de Costa Rica}}\\


    \ThisURCornerWallPaper{0.5}{./images/granosgirasol.png}
 
  \end{center}

\end{titlingpage}

%%% Local Variables:
%%% mode: latex
%%% TeX-master: "plantilla_proyecto"
%%% End:

\mainmatter
\fancyhead[LE,RO]{\slshape \rightmark}
\fancyhead[LO,RE]{\slshape \leftmark}
\lhead{CA-404 Modelos Lineales}
\rhead{\thepage}
\fancyfoot[C]{\thepage}

\tableofcontents*

\newpage
\chapter{Bitácora 1}

\section{Características de la Tabla de datos}

Se conoce a los mercados inmobiliarios como el conjunto de las acciones
de oferta y demanda de bienes inmuebles, no obstante, la promoción
inmobilaria, la inversión que se puede realizar por instituciones y
compañias, así como individuos, y por último la financiación, la
adquisión inmobilaria.

El presente proyecto de investigación tiene como objetivo analizar con
herramientas estadísticas una base de datos cerrada y contextualizada en
los Estados Unidos de América, analizar y predicir hacia donde se están
movilizando los precios de las propiedades. La base de datos fue
encontrada en Kaggle. Esta base toma en cuenta observaciones desde el 2
de mayo del 2014, hasta el 10 de julio del mismo año, sin embargo, la
base fue publicada en el 2018. En total se encuentran 4600
observaciones.

Sobre la base de datos, la población de estudio está determinada las
casas ubicadas en ciudades de los Estados Unidos de América. La muestra
de estudio es el conjunto de datos que contiene una serie de
características que describen los parámetros de las casas ubicadas en
las distintas ciudades de los Estados Unidos. Es decir, la muestra se
refiere al comportamiento del precio de las propiedades. Ante lo
anterior la unidad elemental de estudio son los datos que favorecen el
precio de la propiedad inmobiliaria, observados de manera diaria por los
dos meses de estudio.

El presente compilado de datos posee algunas variables que se procede a
nombrar y explicar:

\begin{itemize}
\item $date:$ La fecha en la cual se obtuvo los datos.
\item $price:$ El precio de la casa, expresado en dólares.
\item $bedrooms:$ Número de dormitorios.
\item $bathrooms:$ Número de baños
\item $sqft_living:$ Superficie habitable.
\item $sqft_lot:$ Tamaño del lote.
\item $floors:$ Número de pisos, entrepisos aceptables.
\item $waterfront:$ Vista al mar, característica dicotómica.
\item $view:$ Vista, en la escala del 0 al 4.
\item $condition:$ Condición de la casa. en la escala del 1 al 5.
\item $sqft_above:$ Superficie de la casa.
\item $sqft_basement:$ Superficie del sótano, en caso de que posean uno.
\item $yr_built:$ Año de construcción.
\item $yr_renovation:$ Año de renovación, en caso de que hubiese.
\item $street:$ Dirección 
\item $city:$ Ciudad 
\item $staetzip:$ Código postal.
\item $country:$ País.
\end{itemize}

*\textbf{Las características del área se expresan en pies cuadrados.}

La pregúnta de la investigación sería:

¿Cuál es el impacto en el precio de las casas, tomando en cuenta el
número de dormitorios, el precio de la casa, número de baños, tamaño del
lote, número de pisos, estado de la casa, año de construcción,
superficie de la casa, año de renovación y la ciudad?

\section{Base técnica UVE}

El objeto de estudio de la investigación sería buscar sentar las bases
estadísticas que permitan comprender el precio de las casas en los
Estados Unidos de América, en el periodo de contrucción de ellas, desde
1900 hasta el 2014.

Algunos conceptos básicos que delimiten teóricamente la pregunta de
investigación serían:

\begin{itemize}
\item Casa: El término significa la construcción de una o pocas plantas que están destinadas a habitarlas familias, caso opuesto con respecto a los edificios de múltiples departamentos o pisos, según la definición de la \citep{RAE}.

\item Precio de vivienda: Según \citet{Bueno}
el precio de una vivienda, el cual no es solo tarifa, si no, que es parte de varios componentes de distintos rango, según el momento del mercado. 

\item Superficie construida: Según la revista \citet{AD}, se le llama de esa forma, a la suma total de los metros cuadrados que están dentrp del perímetro de la casa.

\item Estado físico de vivienda: Según el \citet{INEC}, se define estado físico de la vivienda al estado de los componentes de la casa, como paredes, techo y el piso.

\item Dimensión del terreno: Según \citet{Maria}, el tamaño estánda de este es de 60 y los 84 metros cuadrados de terreno.

\item Valoración de propiedad: Según la \citet{Consultoria} la valoración de bienes muebles o inmuebles es un proceso requerido en diversidad de ámbitos administrativos, empresariales, judiciales, etc.

\item Plusvalía: \citet{Max} lo define como la diferencia entre el valor de uso y el valor de cambio, que resulta en un beneficio para el capitalista y es el motor del capitalismo.

\item Depreciación: La depreciación es la pérdida de valor de un bien como consecuencia de su desgaste con el paso del tiempo. \citep{Roberto}
\end{itemize}

Algunas teorías o principios que respaldan la pregunta de investigación
son:

Precios hedónicos:

Los precios hedónicos consisten en identificar factores de precio
conforme la base de que el precio de un bien se determina por medio de
las características internas del bien vendido y por los factores
externos que lo afectan. En general, suelen usarse para estimar los
valores cuantitativos de los servicios ambientales o ecosistemas que
influyen en los precios del mercado de las viviendas. Estos precios,
identifican los factores y las características tanto internas como
externas que afectan el precio de un producto del mercado

Regresiones aditivas generalizadas (GAM) :

Las regresiones generalizadas aditivas son una extensión tanto del
Modelo Lineal Múltiple como de los Modelos Aditivos, dicho modelo es
propuesto por Hastie y Tibshirani(1986,1990). En el caso de los Modelos
de Regresión Lineal Múltiple se tiene que la media de la variable Y
(variable respuesta), con k covariables está dada por la forma lineal:

\[E[Y|X]= \beta _0 + \beta _1 X_1 + ... + \beta _k X_k\] De forma que,
para los modelos Aditivos, se tienen \(X_i\) covariables, mediante unas
funciones \(h_i\), con el fin de lograr mayor flexibilidad en el modelo
y que sea capaz de adaptarse a datos más complejos, que no presenten
incluso linealidad extricta entre las covariables.Sin embargo se agregan
siempre manteniendo la linealidad, de forma que viene dada por:

\[E[Y|X] = h_0 + h_1(X_1) + ... + h_k (X_k)\] Es importante aclarar que
los modelos aditivos, siempre deben cumplir todas las suposiciones que
piden verificar los modelos de regresión lineal. También la variable Y
debe cumplir con tener distribución normal. De forma que los modelos
aditivos generalizados GAM, generalizan los Modelos Aditivos,
anteriormente mencionados, permitiendo que la variable de respuesta Y
tenga una distribución exponencial. Así este tipo de modelos son buenos
cuando no se cumple el supuesto de tener distribución normal para la
variable Y.

Cabe resaltar que son de los modelos más flexibles ya que también
generalizan a los modelos lineales, solo basta que las funciones \(h_i\)
correspondan a la función identidad y por lo tanto sean todas iguales.

A continuación se presenta el siguiente estudios previo, que respalda y
podría ser de apoyo para la presente investigación:

La investigación se titula Precios de vivienda hedónicos no paramétricos
y se llevó a cabo por Carl Mason y John M.Quigley. En el cual hacen uso
de los precios hedónicos para modelar el precio de la vivienda, sin
embargo este método puede ser complejo ya que los resultados de los
modelos hedónicos dependen de la inclusión de las adecuadas variables
independientes y también de una buena especificación de la forma
funciona. Sin embargo, este último punto suele ser un poco difícil en el
contexto de viviendas debido a que la función de precio hedónico resume,
además de la tecnología de producción y las preferencias del consumidor,
las cantidades históricamente determinadas. Por lo cual, para dicha
investigación se decide estimar la funcióon del precio hedónico por el
Modelo Aditivo General (GAM). Siendo este modelo el que se usará para el
análisis de la disminución sustancial del valor de las viviendas en
condominio en Los Ángeles, en la década de los 80's, años en los cuales
el precop de los apartamentos perdieron aproximadamente un 40\% de su
valor original

\printbibliography

\backmatter
\end{document}
